\documentclass{article}
\usepackage[utf8]{inputenc}
\usepackage[polish]{babel}
\usepackage[T1]{fontenc}
\usepackage{graphicx}
\usepackage{amsmath}
\usepackage{tikz}
\usepackage{mathdots}
\usepackage{yhmath}
\usepackage{cancel}
\usepackage{color}
\usepackage{siunitx}
\usepackage{array}
\usepackage{multirow}
\usepackage{amssymb}
\usepackage{gensymb}
\usepackage{tabularx}
\usepackage{booktabs}
\usepackage{multicol}
\usepackage{comment}
\usetikzlibrary{fadings}
\usetikzlibrary{patterns}
\usetikzlibrary{shadows.blur}
\usetikzlibrary{shapes}


\title{ Zagraj w hokeja z ładunkami elektrycznymi \\
\large Dokumentacja}

\author{Kamil Sikora,
Maciej Ładoś,
Michał Bar}
\date{April 2020}

\begin{document}

\maketitle

\newpage
\tableofcontents

\newpage

\section{Opis projektu}
Projekt polega na symulacji gry w hokeja z dodatnio naładowanym krążkiem poruszanym jedynie poprzez wpływ innych ładunków elektrycznych. Gracz sam decyduje o rozmieszczeniu ładunków. Celem gry jest umieszczenie krążka w bramce. Gra kończy się po strzeleniu gola lub dotknięciu przez krążek któregoś z ładunków bądź przeszkód.\\
Projekt zdecydowaliśmy się wykonać w JavaScript

\section{Model fizyczny}
Fizyka w grze jest oparta na prawie Coulomba, którego treść brzmi: Siła wzajemnego oddziaływania dwóch naładowanych cząstek jest wprost proporcjonalna do iloczynu wartości tych ładunków i odwrotnie proporcjonalna do kwadratu odległości między nimi.

$$F=k \frac{q 1 \cdot q 2}{r^{2}}$$
$F$ - siła elektrostatyczna \\
$q1, q2$ - ładunki elektryczne \\
$r$ - odległość \\
$k$ - stała elektrostatyczna w przybliżeniu równa $9 \cdot 10^9 \frac{N \cdot m^2}{C^2}$
\\

\noindent Oba ładunki mają również masę, lecz w świecie cząstek, atomów i cząsteczek chemicznych, możemy całkowicie zaniedbać ich wzajemne oddziaływania grawitacyjne.


\begin{figure}[h]
    \centering
    \includegraphics[scale=0.5]{img/Prawo_Coulomba.PNG}

    \caption{Prawo Coulomba- oddziaływanie ładunków}
    \label{fig:Prawo_Coulomba}
\end{figure}

\clearpage

\section{Model symulacyjny}

Nasza symulacja operuje w płaszczyźnie dwuwymiarowej. Każdy obiekt będzie posiadał dwie współrzędne - $x$ i $y$. Odległość między ciałami może być zatem wyznaczona jako odległość punktów na płaszczyźnie.

$$
r=\sqrt{\left(x_{b}-x_{a}\right)^{2}+\left(y_{b}-y_{a}\right)^{2}}
$$

\noindent W celu uproszczenia obliczeń siłę $\vec{F}$ oddziałującą na ładunki rozkładamy na składowe $\vec{F_x}$ oraz $\vec{F_y}$. Zakładamy, że środkiem naszego układu współrzędnych będzie krążek -- pozwoli nam to na skorzystanie z funkcji trygonometrycznych do wyliczenia tych składowych.

\tikzset{every picture/.style={line width=0.75pt}} %set default line width to 0.75pt

\tikzset{every picture/.style={line width=0.75pt}} %set default line width to 0.75pt        

\begin{center}
\begin{tikzpicture}[x=0.75pt,y=0.75pt,yscale=-1,xscale=1]
%uncomment if require: \path (0,261); %set diagram left start at 0, and has height of 261


%Shape: Right Triangle [id:dp36130500270564525] 
\draw   (387,62) -- (521.5,219) -- (387,219) -- cycle ;
%Straight Lines [id:da4371175600607653] 
\draw    (387,62) -- (387,216) ;
\draw [shift={(387,219)}, rotate = 270] [fill={rgb, 255:red, 0; green, 0; blue, 0 }  ][line width=0.08]  [draw opacity=0] (8.93,-4.29) -- (0,0) -- (8.93,4.29) -- cycle    ;
%Straight Lines [id:da010907474739860756] 
\draw  [dash pattern={on 4.5pt off 4.5pt}]  (387,62) -- (519.5,62.42) ;
\draw [shift={(522.5,62.43)}, rotate = 180.18] [fill={rgb, 255:red, 0; green, 0; blue, 0 }  ][line width=0.08]  [draw opacity=0] (8.93,-4.29) -- (0,0) -- (8.93,4.29) -- cycle    ;
%Straight Lines [id:da31420475263437164] 
\draw    (387,62) -- (519.55,216.72) ;
\draw [shift={(521.5,219)}, rotate = 229.41] [fill={rgb, 255:red, 0; green, 0; blue, 0 }  ][line width=0.08]  [draw opacity=0] (8.93,-4.29) -- (0,0) -- (8.93,4.29) -- cycle    ;

% Text Node
\draw (463,115) node [anchor=north west][inner sep=0.75pt]   [align=left] {$\displaystyle r$};
% Text Node
\draw (383,38) node [anchor=north west][inner sep=0.75pt]   [align=left] {$\displaystyle k$};
% Text Node
\draw (523.5,222) node [anchor=north west][inner sep=0.75pt]   [align=left] {$\displaystyle s$};
% Text Node
\draw (448,228) node [anchor=north west][inner sep=0.75pt]   [align=left] {$\displaystyle y$};
% Text Node
\draw (365,132.4) node [anchor=north west][inner sep=0.75pt]    {$x$};
% Text Node
\draw (361,181.4) node [anchor=north west][inner sep=0.75pt]    {$\vec{F}_{x}$};
% Text Node
\draw (493,30.4) node [anchor=north west][inner sep=0.75pt]    {$\vec{F}_{y}$};
% Text Node
\draw (411,65) node [anchor=north west][inner sep=0.75pt]   [align=left] {$\displaystyle \alpha $};
% Text Node
\draw (519,182.4) node [anchor=north west][inner sep=0.75pt]    {$\vec{F}$};

\end{tikzpicture}
\end{center}

\noindent Korzystając z funkcji trygonometrycznych otrzymamy:

$$
\sin \alpha=\frac{\vec{F}_{x}}{r}=\frac{x}{r} \quad \quad \cos \alpha=\frac{\vec{F}_{y}}{r}=\frac{y}{r}
$$

\noindent Przyjmujemy, że wartości ładunków są takie same, więc pomijamy je w obliczeniach. Wyprowadzamy wzory na składowe:

$$
\vec{F}_{x}=\frac{\sin \alpha}{r^{2}} \quad \vec{F}_{y}=\frac{\cos \alpha}{r^{2}}
$$

\noindent Korzystając z II zasady dynamiki Newtona, wyprowadzamy wzór na przyspieszenie ładunku:

$$
F=m \cdot a \Rightarrow a=\frac{F}{m}
$$

\noindent Ostatecznie wzory:

$$
a_{x}=k \cdot \frac{\sin \alpha}{m \cdot r^{2}} \quad a_{y}=k \cdot \frac{\cos \alpha}{m \cdot r^{2}}
$$


\section{Implementacja}
\subsection{Struktura kodu}
\begin{enumerate}
    \item const - folder zawiera pliki przechowujące wszelkie stałe wykorzystywane w programie
    \item assets - folder z plikami graficznymi oraz plikiem css
    \item controllers - są tu zawarte klasy obsługujące interakcje klienta z programem oraz akcje wykonywane przez program
    \item events - zawiera klasę która zapewnia obsługę zdarzeń w grze
    \item extensions - folder z rozszerzeniami wykorzystywanymi w programie
    \item models - zawarte są tu wszelkie klasy tworzące obiekty w grze oraz podfolder physics, który przechowuje obiekty fizyczne oraz model fizyczny
    \test testy
\end{enumerate}
\subsection{Wybrane stałe}
\begin{enumerate}
    \item COULOMB\_FORCE\_FACTOR - odpowiada stałej k z obliczeń z sekcji 'Model Symulacyjny'
    \item CHARGE\_MIN\_DISTANCE - minimalna odległość na jakiej utrzymywane są ładunki, aby nie dopuścić do współistnienia krążka i ładunku w tym samym miejscu i czasie zgodnie z zasadą Pauliego. W związku z tym zakładamy w naszym modelu dystans między ładunkami na minimum 25 jednostek odległości
    \item PUCK\_VELOCITY\_DIVIDER - podzielnik prędkości aby wizualizacja symulacji była lepiej dostrzegalna
\end{enumerate}

\subsection{Wybrane modele i ich metody}
\begin{enumerate}
    \item BoundingBox - główna klasa zapewniająca ruch oraz kolizje obiektów na planszy.
    Jej pola to \textit{x,y} składowe położenia oraz szerokość \textit{width} i wysokość \textit{height} obiektu.
    
    Zawiera trzy metody do obsługi kolizji: \textit{touches(), contains(), intersects()}.
    
    \textit{move()} przypisuje nowe położenie obiektu na podstawie przekazanych parametrów dotyczących przemieszczenia.
    
    \textit{distance()} oblicza dystans między obiektami w grze.
    
    \item GameObject - rozszerza \emph{BoundingBox}, bazowa abstrakcyjna klasa dla wielu modeli w grze. Deklaruje metody \textit{update(), render()}.
    \item Group - klasa implementująca działania na tablicach obektów tworzonych jako grupy w grze
    \item HockeyGoal - klasa implementująca rysowanie bramki na podstawie tworzonego \emph{BoundingBoxa}
    \item Obstacle - klasa dzięki która tworzy i renderuje przeszkody w grze
    \item Trace - klasa wykorzystywana przez klasę \textit{Puck} do rysowania trasy krążka
    \item ElectricCharge - bazowa klasa dla klas \textit{NegativeCharge, PositiveCharge} tworząca odpowiedni ładunek na podstawie przekazanego typu \emph{type}.\\
    Typ jest zawarty domyślnie w konstruktorach klas \textit{NegativeCharge, PositiveCharge}.
    \item Puck - klasa dziedzicząca po \emph{ElectricCharge} tworząca krążek jako ładunek ujemny lub dodatni.\\
    Krążek posiada własną masę \emph{mass} oraz promień \emph{radius}.
    
    \emph{acceleration, velocity} to odpowiednio przyspieszenie i prędkość krążka potrzebne do wyznaczania jego ruchu.
    
    \emph{trace} gdzie przypisany jest obiekt klasy \emph{Trace}  oraz \emph{traceIsActive} dotyczą możliwości rysowania przebytej przez krążek trasy.
    
   \textit{update(), render()} zapewniają odpowiednio ruch oraz rysowanie krążka.
   
   \item CoulumbForce - klasa implementująca obliczenia modelu fizycznego. Oblicza przyspieszenie krążka na podstawie siły między dwoma przekazanymi ładunkami. Jej obiekt jest tworzony podczas dodawania ładunku w \emph{GameController} i dodawany do \emph{forces} w klasie \emph{Game}
  
    \begin{figure}[h]
    \centering
    \includegraphics[scale=0.8]{img/Funkcja_calculate.PNG}

    \caption{Funkcja calculate obliczająca przyspieszenie krążka na podstawie siły między ładunkami}
    \label{fig:Funkcja_calculate}
\end{figure}

\emph{calculate} - główna funkcja obliczająca przyspieszenie wypadkowe krążka, wykorzystywana w \emph{updateForces()} w \emph{Game}

\emph{displacement} przechowuje przemieszczenie między krążkiem a ładunkiem dla danych składowych

\emph{r} odległość między krążkiem a ładunkiem

\emph{rCube} odległość podniesiona do sześcianu, upraszcza obliczenia w\\
\emph{calculateComposite}

\emph{x,y} składowe wyznaczanej siły

    \begin{figure}[h]
    \centering
    \includegraphics[scale=1.2]{img/calculateComposite.PNG}

    \caption{pomocnicza funkcja wyznaczająca przyspieszenie}
    \label{fig:Funkcja_calculate_composite}
\end{figure}

\emph{calculateComposite()} funkcja oblicza przyspieszenie dla podanego przemieszczenia składowego. 
\\Odpowiednik w modelu fizycznym $$
a_{x}=k \cdot \frac{\sin \alpha}{m \cdot r^{2}} \quad a_{y}=k \cdot \frac{\cos \alpha}{m \cdot r^{2}}
$$
Odpowiednikiem sinusa lub cosinusa jest $\frac{\emph{displacement}}{r}$. Jako że w mianowniku występuje również $r^2$, wyrażenie $\frac{\emph{displacement}}{r \cdot r^2}$ zostało zastąpione przez $\frac{\emph{displacement}}{rCube}$

Czynniki \emph{charge.getSign()} w liczniku odpowiadają za wyznaczenie odpowiedniego kierunku przyspieszenia
    
    
\end{enumerate}

\subsection{Kontrollery}
\begin{enumerate}
    \item Controller - bazowa klasa dla pozostałych kontrolerów zawierająca pole \emph{eventBus} przechowujące obiekt klasy \emph{Eventbus} do obsługi zdarzeń w grze.
    
    \item InputController - klasa obsługująca zdarzenia wymuszone przez klienta przez funkcjonalności takie jak: przycisk, checkbox, pole typu 'input' i kliknęcia myszą. Dla każdego typu interakcji użytkownika z programem jest zaimplementowana jedna z funkcji \textit{registerButtonListeners(), registerCheckboxListeners(), registerRadioListeners(), registerInputListeners(), registerMouseListeners()} wysyłąjąca odpowiedni komunikat dzięki \emph{eventBus.emit()}.
    \item GameControler - klasa obsługująca wszelkie zdarzenia w grze.
    
    \emph{game} pole przechowujące aktualną sesję gry.
    
    \emph{registerListeners()} główna funkcja implementująca reakcję na konkretne zdarzenie wysłane przez grę lub użytkownika poprzez \emph{InputController}. Przez \emph{eventBus.on()} sprawdza czy aktywne jest dane zdarzenie
    
    \emph{clear()} wyczyszczenie planszy i stanu gry
    
    \emph{onDifficultyChange()} - funkcja reagująca na zmianę poziomu trudności
    
    \emph{placeCharge()} - funkcja obsługująca umiejscowienie ładunków na planszy
    
    \emph{showGoalMessageAnimation(), showFailureMessageAnimation()} - funkcje wyświetlające animacje tekstu w zależności od zdarzenia
    \end{enumerate}



\section{Literatura}
\begin{enumerate}
    \item Zbigniew Kąkol, Kamil Kutorasiński. Prawo Coulomba. 
    \item Zbigniew Kąkol. Fizyka. Kraków 2019
    \item David Halliday, Robert Resnick, Jearl Walker. Podstawy fizyki. Tom 3
\end{enumerate}

\end{document}

